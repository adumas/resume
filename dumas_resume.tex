%!TEX TS-program = xelatex
%!TEX encoding = UTF-8 Unicode

\documentclass[letterpaper,11pt]{article}

%A Few Useful Packages
\usepackage{marvosym}
\usepackage{fontspec} 					%for loading fonts
\usepackage{xunicode,xltxtra,url,parskip} 	%other packages for formatting
\RequirePackage{color,graphicx}
\usepackage[usenames,dvipsnames]{xcolor}
%\usepackage[big]{layaureo}
%\usepackage{fullpage}				%better formatting of the A4 page
\usepackage[left=1.3cm,top=0.8cm,right=1.3cm,bottom=0.8cm,nohead,nofoot]{geometry}
% an alternative to Layaureo can be ** \usepackage{fullpage} **
\usepackage{tabularx}
\usepackage{titlesec}					%custom \section
\usepackage{paralist}
\usepackage{multicol}

%Setup hyperref package, and colours for links
\usepackage[bookmarks, colorlinks, breaklinks, pdftitle={Andrew Dumas Resume},pdfauthor={Andrew P. Dumas}]{hyperref}
\definecolor{linkcolour}{rgb}{0,0,0}
\hypersetup{colorlinks,breaklinks,urlcolor=linkcolour, linkcolor=linkcolour}

%FONTS
\defaultfontfeatures{Mapping=tex-text} % converts LaTeX specials (``quotes'' --- dashes etc.) to unicode
\setromanfont [Ligatures={Common}, Numbers={OldStyle}]{Hoefler Text}
\setmonofont[Scale=0.8]{Monaco}
\setsansfont[Scale=0.9]{Optima Regular}
% ---- CUSTOM AMPERSAND
\newcommand{\amper}{{\fontspec[Scale=.95]{Hoefler Text}\selectfont\itshape\&}}

%CV Sections inspired by:
%http://stefano.italians.nl/archives/26
\titleformat{\section}{\Large\scshape\raggedright}{}{0em}{}[\titlerule]
\titlespacing{\section}{0pt}{3pt}{3pt}
%Tweak a bit the top margin
%\addtolength{\voffset}{-0.3cm}


%-------------WATERMARK TEST [**not part of a CV**]---------------
\usepackage[absolute]{textpos}

\setlength{\TPHorizModule}{30mm}
\setlength{\TPVertModule}{\TPHorizModule}
\textblockorigin{2mm}{0.65\paperheight}
\setlength{\parindent}{0pt}

%--------------------BEGIN DOCUMENT----------------------
\begin{document}



\pagestyle{empty} % non-numbered pages

\font\fb=''[cmr10]'' %for use with \LaTeX command

%--------------------TITLE-------------
\par{\centering
		{\huge Andrew P. \textsc{Dumas}
	}\par}

%--------------------SECTIONS-----------------------------------
%Section: Contact Info
\begin{flushright}
\begin{tabular}{rl}
    \textsc{address:}   & 31 Cherry Street, Apt.\#3 \textbullet\ Somerville, MA 02144 \\
    \textsc{cell:}     & (508)-631-6216\\
    \textsc{email:}     & \href{mailto:apdumas@gmail.com}{apdumas@gmail.com}
\end{tabular}
\end{flushright}



%Section: Work Experience
\vspace{-0.2cm}
\section{Work Experience}
\begin{tabular}{r|p{14.5cm}}
%% Want to make dates on two lines to make first column tighter ####
    \textsc{July 2012-present} & Signal Processing Engineer, \textsc{MIT Lincoln Laboratory}, Lexington, MA \\&\emph{Bioengineering Systems and Technologies}\\&\footnotesize{Develop algorithms for gait analysis using wearable inertial sensors. Data analysis on a wide range of biomedical projects involving real-time MRI of speech, auditory physiology, gait during load carriage, and thermal heat strain. }\\\multicolumn{2}{c}{} \vspace{-0.2cm}\\
\textsc{July 2010-Jul. '12} & Research Technologist, \textsc{Massachusetts General Hospital}, Charlestown, MA \\&\emph{Hemorrhagic Stroke Research Program and Athinoula A. Martinos Center for Biomedical Imaging}\\&\footnotesize{Conducted research investigating decreased vascular reactivity in Cerebral Amyloid Angiopathy using functional MRI to model hemodynamic response. Implemented algorithms in MATLAB for non-linear curve fitting, general linear modeling, and image processing.}\\\multicolumn{2}{c}{} \vspace{-0.2cm}\\
\textsc{June 2010-Aug. '10} & Research Student, \textsc{Beth Israel Deaconess Medical Center}, Boston, MA \\&\emph{Cardiac MRI Department}\\&\footnotesize{Created and tested algorithms to map T1 in phantoms and in human cardiac tissue, with emphasis on algorithm speed and robustness as well as rapid imaging acquisition time.}\\\multicolumn{2}{c}{} \vspace{-0.2cm} \\
%\textsc{June 2008-May '09} & IT Staff Assistant, \textsc{Boston University}, Boston, MA \\&\emph{ Department of Electrical and Computer Engineering}\\&\footnotesize{Set up and managed three instructional computing labs (up to 120 workstations and 10 networked printers). Developed multiple silent software installations over the network using Group Policy, and shell scripting.}\\
\end{tabular}

%% ADD PUBLICATIONS SECTION ####

%Section: Education
\vspace{-0.2cm}
%%Want to move everything together towards the middle ####
\section{Education}
\hspace{5mm}
\begin{tabular*}{0.9\textwidth}{@{\extracolsep{\fill}} ll}
MA, \textsc{Biomedical Imaging} & \hfill \textbf{Boston University}, Boston, MA\\
\hspace{2mm} Graduated \textsc{Aug.} 2010 \normalsize \hspace{3mm}  & \hspace{2mm} \emph{School of Graduate Medical Sciences}\\
BS, \textsc{Biomedical Engineering } & \hfill \normalsize\textbf{Boston University}, Boston, MA\\
\hspace{2mm} Graduated \textsc{May} 2009 \normalsize  \hspace{3mm} & \hspace{2mm} \emph{College of Engineering} \\
\hspace{3mm}{\normalsize Member, \textsl{Alpha Eta Mu Beta} (Biomedical Engineer Honor Society)}
\end{tabular*}

%Section: Relevant Coursework
\vspace{-0.2cm}
 \section{Relevant Coursework}
\begin{center}
\footnotesize
\begin{tabular}{p{0.3\textwidth}p{0.3\textwidth}p{0.3\textwidth}}
Applied Bioinformatics & Biological \&  Environmental Acoustics & Biomedical Signal Measurement\\
Control Systems & Engineering Economics & Head and Neck Anatomy\\
Imaging Theory \& Image Processing & Logic Design using Verilog & Methods of Functional Neuroimaging\\
Signals and Systems & Solid Biomechanics & Intellectual Assets\\
\end{tabular}
\end{center}


%Section: Project Experience
\vspace{-0.2cm}
\section{Project Experience}
\begin{tabular}{r|p{14.5cm}}
\textsc{Sep. '09-Aug. '10} & \small{\emph{Development of an Offline Tool for Susceptibility Weighted Image (SWI) Processing using \textsc{Matlab}}} \\
\textsl{MA Thesis} & \footnotesize{Developed a tool using \textsc{Matlab} to process MRI phase and magnitude images and output images with susceptibility weighted contrast. Tested the algorithm using human brain images acquired with custom SWI (T2*-weighted) sequences. Created a GUI to facilitate offline processing.}\\
\textsc{Sep. '08-May '10} & \small{\emph{Simulating Echolocation using Computational Models of Auditory Physiology}} \\
\textsl{Senior Project} & \footnotesize{Collected ultrasonic echoes in response to a synthetic ``chirp'' characterizing common obstacles (such as chairs, tables, walls, etc.) using an ultrasonic emitter and binaural detectors to mimic echolocation. Developed algorithms to estimate object distance and azimuth and to classify object based on previous data.}
\end{tabular}

%Section: Publications
%\vspace{-0.2cm}
%\section{Publications \amper \ Presentations}
%A.~Dumas, G.~A. Dierksen, M.~E.~Gurol, et al, (in-press), ``Functional MRI detection of vascular reactivity in cerebral amyloid angiopathy,'' {\em Annals of Neurology}, 2012.\\
%\textsc{2011} Annual Massachusetts ADRC \& Boston University ADC Poster Session\\
%\textsc{2011} & International Stroke Conference, Guided Poster Session\\


%Section: Computer Skills
\vspace{-0.3cm}
\section{\texorpdfstring{Computer \amper \ Technical Skills}{Computer and Technical Skills}}
\begin{tabularx}{\textwidth}{rX}
\textsc{Programming:} & Matlab (GUI development, statistical analysis, machine learning, visualization), Mathematica, Perl, BASH shell scripting\\
%\textsc{Electronics:} & Arduino microcontroller programming, breadboard circuit prototyping, electronics troubleshooting, oscilloscope operation\\
\textsc{Imaging:} & MR Spectroscopy, fMRI Acquisition, FreeSurfer, FSL, SPM8, Siemens and Philips MRI Scanner operation, MRI Magnet Safety, NIH Human Subjects Certification\\
\textsc{General Computing:} & Linux/Unix, Subversion, Git
\end{tabularx}




\end{document}

%EXTRA EXTRA EXTRA:

%Section: OBJECTIVE
\vspace{-0.5cm}
\section{Objective}
To obtain a job developing innovative medical devices utilizing my experience in the fields of biomedical engineering and imaging involving analog and digital electronics design, leveraging my background in Matlab as well as signal and image processing knowledge, and my strong desire to learn new skills.

%from project experience:
 \textsc{Sep '08-May '09} & \small{\emph{“Simulating Echolocation using Computational Models of Auditory Physiology”}}  \\
 \textsl{Senior Project} &\footnotesize{\textbf{Description:} The aim of this project was to obtain data using an ultrasonic emitter to mimic echolocation and to process the signals using existing auditory models. The simulation estimated target range and azimuth relative to the sound source, and attempted to identify the object using a spectrogram correlation algorithm.}\\\multicolumn{2}{c}{} \\


 Applied Bioinformatics & Biological \&  Environmental Acoustics\\
Biomedical Measurement Lab & Control Systems \\
Head and Neck Anatomy & Imaging Theory \& Image Processing\\
Intellectual Assets: Creation, Protection, and Commercialization & Logic Design\\
Methods of Functional Neuroimaging & Signals and Systems\\
Solid Biomechanics

%WATERMARK TEST [**not part of a CV**]---------------
\font\wm=''Baskerville:color=787878'' at 8pt
\font\wmweb=''Baskerville:color=FF1493'' at 8pt
{\wm
	\begin{textblock}{1}(0,0)
		\rotatebox{-90}{\parbox{500mm}{
			Typeset by Andrew Dumas with \XeTeX\  \today\
		}
	}
	\end{textblock}
}
